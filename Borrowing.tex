% !TeX encoding = UTF-8
% !TeX spellcheck = de_DE
% !TeX root = ./mainDoc.tex

\section{Method Borrowing}

JavaScript macht es der Programmiererin sehr leicht, eine bestimmte Methode eines Objekts in einem anderen Kontext (das bedeutet mit einem anderen \texttt{this}-Binding) auszuführen. Damit kann eine für ein Objekt entwickelte Methode auf ein anderes Objekt angewendet und so wiederverwendet werden. 

Nach den im vorigen Abschnitt erläuterten Regeln ist eine Methode normalerweise mit dem Objekt, auf dem sie aufgerufen wird, über das implizite \texttt{this} verbunden. JavaScript bietet die eingebauten Funktionen \texttt{call()} und \texttt{apply()} an, mit denen sich das \texttt{this}-Binding bei einem Funktionsaufruf explizit überschreiben lässt. \texttt{call()} und \texttt{apply()} unterscheiden sich lediglich darin, wie die Parameter des Funktionsaufrufs übergeben werden. Bei \texttt{call()} werden sie als Parameterliste angegeben während sie bei \texttt{apply()} als Array übergeben werden.
Dies wird an einem Beispiel deutlich:

\insertcode{codesnips/callBorrowing.js}{Die Methode \texttt{doStuff()} des Objekts \texttt{notmyobj} wird aufgerufen und explizit per \texttt{call(myobj)} bzw. \texttt{call(myobj)} an ein anderes Objekt gebunden.
(Beispiel aus \citep{StefanovJavaScriptpatternsbuild2010})}.

Man spricht bei der gezeigten Technik von \emph{method borrowing}, da sich ein Objekt eine Methode eines anderen Objekts ausleiht.

Diese Technik wird sehr häufig angewendet, um Funktionen eingebauter Objekte,  z.~B. des \texttt{Array}-Objekts zu benutzen, obwohl das Objekt, auf dem operiert wird, selber kein Array, sondern nur \emph{array-like} ist. Die innerhalb von Funktionen belegte Variable \texttt{arguments} ist ein solches array-like Objekt (siehe \citep{MozillaDeveloperNetworkargumentsobject}). Um es in ein \emph{echtes} Array umzuwandeln, kann die \texttt{slice()} Methode aus \texttt{Array.prototype} ausgeliehen werden:
\insertcode{codesnips/toArray.js}{Die Methode \texttt{Array.prototype.slice} wird ausgeliehen und auf das array-like \texttt{arguments} Objekt angewendet.}

Wenn eine Funktion in einer Variable gespeichert werden soll oder als Callback übergeben und später aus einem anderen Kontext heraus aufgerufen wird, so verliert sie ihr \texttt{this}-Binding. \texttt{this} fällt beim Aufruf auf das default-Binding zurück, das im non-strict-Mode auf das \emph{global Object} zeigt und im strict-Mode zu einem Laufzeitfehler führt. Um eine Funktion fest an einen Kontext zu binden, gibt es seit ES5 die Funktion \texttt{Function.prototype.bind()}, die es erlaubt ein festes Objekt als \texttt{this}-Binding und optional weitere Funktionsparameter zu fixieren.
\insertcode{codesnips/bindExample.js}{\texttt{bind()} zur festen \texttt{this}-Bindung (Beispiel frei nach \citep{StefanovJavaScriptpatternsbuild2010})}

Die aus dem Objekt \texttt{one} geborgte Methode wird zunächst ungebunden in eine Variable gespeichert (l.~12), so dass ihr \texttt{this} beim Aufruf auf das globale Objekt zeigt, das keine Property \texttt{name} besitzt. Per \texttt{call()} kann das \texttt{this}-binding explizit gesetzt werden (l.~14). Optional kann auch die Funktion vor dem Speichern in einer Variablen fest an ein bestimmtes Objekt gebunden (l.~17) und zusätzlich mit festen Funktionsparametern (l.21) versehen werden. Eine derart fest gebundene Funktion kann auch problemlos als Parameter für Callbacks verwendet werden.



%
%
%Wenn einfach eine funktion wieder verwendet werden soll, so macht es JavaScript der Programmiererin leicht:
%In JS kann jede Funktion mit einem beliebigen Kontext aufgerufen werden. Der Kontext ist dabei das \texttt{this}-Binding. Üblicherweise ist das \texttt{this}-Binding über eine der oben genannten Regeln festgelegt.
%
%Es kann jederzeit ein anderer Kontext per explicit-Binding festgelegt werden. Damit können vorhandene Funktionen eines anderen Objekts ausgeliehen werden.
%
%Sehr üblich ist das bei eingebauten Funktionen, die aber auf dem aktuellen Objekt nicht über die Prototype-Chain ereichbar sind.
%
%Beispiel Array.prototype.slice ist auf \texttt{arguments} nicht verfügbar, da \texttt{arguments}, zwar ein Array-like Object ist, nicht auf die Array-funktionen aus \texttt{Array.prototype} zugreifen kann. (\citep{argumentsobject})
%
%Das geht auch bei eigenen Funktionen und Objekten. Hier ein Beispiel aus \citep{BorrowingMethodsJavaScript}:
%
%
%Anstelle von Borrowing ist es bei eigenen Funktionen besser ein Modul anzulegen und die Funktion von dort zu importieren. \todoCite{\citep[p.86]{ElliottProgrammingJavaScriptapplications2014}}
%

